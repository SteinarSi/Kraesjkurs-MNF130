\subsection{Counting}
\begin{frame}
\begin{block}{Produktregelen}
\begin{itemize}
\item Noe kan brytes ned i to aksjoner som kombineres med hverandre
\item For den ene finnes det $n_1$ muligheter, for den andre $n_2$
\item Det blir $n_1\cdot n_2$ kombinasjoner
\end{itemize}
\end{block}
\pause
\begin{block}{Eksempel}
\begin{itemize}
\item Det er to type maskiner som trenges
\item Den ene finnes 3 ganger, den andre 5 ganger
\item Hvor mange kombinasjoner maskintype 1, maskintype 2 finnes det?
\item $5\cdot 3=15$
\end{itemize}
\end{block}
\end{frame}

\begin{frame}
\begin{block}{Sumregelen}
\begin{itemize}
\item Noe kan gjøres på enten en av $n_1$ måter, eller en av $n_2$ måter
\item Det finnes ingen element som er både i $n_1$ og i $n_2$
\item Det blir $n_1+n_2$ muligheter å gjøre det
\end{itemize}
\end{block}
\pause
\begin{block}{Eksempel}
\begin{itemize}
\item En student skal velge masteroppgaven sin
\item Hun liker tre fagområder
\item I område $n_1$ finnes det 5 temaer, i $n_2$ 3 temaer, i område $n_3$ er det 8
\item Det er $5+3+8=16$ temaer å velge fra
\end{itemize}
\end{block}
\end{frame}

\begin{frame}
\begin{block}{Substraksjonsregelen}
\begin{itemize}
\item Ligner \textit{Sumregelen}, men flere elementer er i flere grupper
\item Noe kan gjøres på enten en av $n_1$ måter, eller en av $n_2$ måter, men noen er i både $n_1$ og $n_2$
\item Det blir $n_1+n_2-felles(n_1,n_2)$ muligheter
\end{itemize}
\end{block}
\pause
\begin{block}{Eksempel}
\begin{itemize}
\item På fredag er det amerikansk-norsk folkefest
\item Det er 22 amerikanere og 18 nordmenn som meldte seg på
\item 3 av dem er både norsk og amerikansk
\item Det er $22+18-3=37$ personer som deltar
\end{itemize}
\end{block}
\end{frame}

\begin{frame}
\begin{block}{Divisjonsregelen}
\begin{itemize}
\item Det er $n$ måter å gjøre noe, men egentlig finnes det for hver måte minst $d$ lignende måter
\item Det blir da $n/d$ forskjellige muligheter
\end{itemize}
\end{block}
\pause
\begin{block}{Eksempel}
\begin{itemize}
\item I en fornøyelsespark for katter blir det talt 400 bein \item Mennesker og andre dyr har ikke lov å være i fornøyelsesparken
\item Hver katt har eksakt fire bein
\item Det betyr det er $400/4=100$ katter
\end{itemize}
\end{block}
\end{frame}

\begin{frame}[fragile]{Permutasjon? Kombinasjon? Variasjon? Hæ?}
\adjustbox{scale=0.9}{
\begin{tikzcd}
                               &                                                         & \text{Blir alle elementer med?} \arrow[ld, "ja"'] \arrow[rd, "nei"] &                                                                  &              \\
                               & \text{Permutasjon} \arrow[ld, "med"'] \arrow[d, "uten"] &                                                                    & \text{Er rekkefølgen viktig?} \arrow[ld, "ja"'] \arrow[d, "nei"] &              \\
 \frac{n!}{r!\cdot s!\cdot t!} & n!                                                      & \text{Variasjon} \arrow[ld, "med"'] \arrow[d, "uten"]              & \text{Kombinasjon} \arrow[d, "med"'] \arrow[rd, "uten"]          &              \\
                               & n^k                                                     & \frac{n!}{(n-k)!}                                                  & \binom{(n+k-1)}{(k-1)}                                           & \binom{n}{k}
\end{tikzcd}
}
\end{frame}


\begin{frame}{Eksempler}
\begin{block}{Eksempel 1}
\begin{itemize}
\item 6 personer må fotograferes. Hvor mange kombinasjoner finnes det å ordne dem på bildet?
\item Alle elementer blir med, ingen repetisjon 
\item Permutasjon uten repetisjon: $n!=6!=720$
\end{itemize}
\end{block}
\pause
\begin{block}{Eksempel 2}
\begin{itemize}
\item Hvor mange måter finnes det for å ordne bokstavene \textit{Mississippi}?
\item Alle elementer blir med, men repetisjon (permutasjon)
\item $\frac{n!}{r!\cdot s!\cdot t!} =  \frac{11!}{4!\cdot 4!\cdot 2!}=34650$
\end{itemize}
\end{block}
\end{frame}

\begin{frame}{Enda flere eksempler}
\begin{block}{Eksempel 3}
\begin{itemize}
\item Det er 7 personer og 3 tilfeldige av dem skal få en pris. Hvor mange kombinasjoner finnes det?
\item Ikke alle elementer blir med, ingen repetisjon, rekkefølgen ikke viktig 
\item Kombinasjon uten repetisjon: $\binom{n}{k}=\binom{7}{3}=\frac{7!}{3!\cdot 4!}=35$
\end{itemize}
\end{block}
\pause
\begin{block}{Eksempel 4}
\begin{itemize}
\item Vi har 5 typer is og skal spise 3 av dem. Vi er opptatt av rekkefølgen for best smak.
\item Ikke alle elementer blir med, rekkefølge viktig, med repetisjon
\item $n^k=5^3=125$
\end{itemize}
\end{block}
\end{frame}
