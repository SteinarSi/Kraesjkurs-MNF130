

\subsection{Sannsynligheter}
\begin{frame}{Definisjoner}
\begin{itemize}[<+->]
\item \textbf{Utfallsrom: }Alle mengder som har en sannsynlighet
\item \textbf{S: }Alle mulige utfall
\item \textbf{E: }Alle ønskete utfall
\item $E \subseteq S$: alle ønskete utfall er en del av alle mulige utfall
\item \textbf{Sannsynlighet: }$P(E)=\frac{|E|}{|S|}$
\item $0 \leq P(E) \leq 1$
\item $P(S)=1$
\item $P(\overline{E})=1-P(E)$
\end{itemize}
\end{frame}

\begin{frame}{Regneregler}
\begin{itemize}[<+->]
\item $P(A\cup B)=P(A)+P(B)-P(A\cap B)$ (tilsvarer subtraction rule)
\item $P(A\cap B)=P(A)\cdot P(B)$ hvis A,B statistisk uavhengig (tilsvarer multiplication rule)
\item $\sum_{s\in S} P(s) = 1 $ (summen av alle ting som kan skje har sannsynlighet 1)
\item \textbf{Betinget sannsynlighet: }$P(A|B)=\frac{P(A\cap B)}{P(B)}$ (Sannsynlighet av A etter at B skjedde)
\item \textbf{Bayes Rule: }$P(A|B)=\frac{P(B|A)\cdot P(A)}{P(B)}$
\end{itemize}
\end{frame}

\begin{frame}{Eksempler}
\begin{block}{Terninger}
\begin{itemize}[<+->]
\item Det er to terninger, den ene er vanlig med 6 jevne sider
\item Den andre har tallene $\{3,4,5,5,6,6\}$
\item Hvor stor er sannsynligheten at vi kaster en 7 med begge terninger?
\item Det er $6\cdot 6$ mulige kombinasjoner
\item Det er fire måter å få det til:
\begin{itemize}
\item Terning 1: 1, Terning 2: 6 (finnes to ganger)
\item Terning 1: 2, Terning 2: 5 (finnes to ganger)
\item Terning 1: 3, Terning 2: 4
\item Terning 1: 4, Terning 2: 3
\end{itemize}
\item $P(sum=7)=\frac{6}{36}=\frac{1}{6}$
\end{itemize}
\end{block}
\end{frame}

\begin{frame}{Eksempler}
\begin{block}{Kortspill}
\begin{itemize}[<+->]
\item Vi har et vanlig tysk kortspill med 32 kort (4 farger, $\{7,8,9,10,J,Q,K,A\}$)
\item Hva er sannsynligheten at vi trekker et hjerte eller en konge?
\item Hva er sannsynligheten at en av disse kortene er 7,8 eller 9?
\item Det er 4 konger, 8 hjerter, en av dem er begge deler
\item $P(Konge\cup Hjerte)=\frac{8+4-1}{32}=\frac{11}{32}$
\item Blant disse er det 3 kort som er 7,8 eller 9
\item $P(7,8,9|Konge\cup Hjerte)=\frac{P((7,8,9)\cap(Konge\cup Hjerte))}{P(Konge\cup Hjerte)}=\frac{3}{32}\cdot \frac{32}{11}=\frac{3}{11}$
\end{itemize}
\end{block}
\end{frame}

\begin{frame}{\textit{Vierfeldertafel} (WANTED: norsk eller engelsk begrep)}
\begin{table}[h!]
\centering
\label{tab:prob_two_events}
\begin{tabular}{l|ll|l}
\cline{2-3}
                            & $A$            & $\overline{A}$            &                               \\ \hline
\multicolumn{1}{|l|}{$B$}   & $P(A\cap B)$   & $P(\overline{A}\cap B)$   & \multicolumn{1}{l|}{$P(B)$}   \\
\multicolumn{1}{|l|}{$\overline{B}$} & $P(A\cap \overline{B})$ & $P(\overline{A}\cap \overline{B})$ & \multicolumn{1}{l|}{$P(\overline{B})$} \\ \hline
                            & $P(A)$         & $P(\overline{A})$         & \multicolumn{1}{l|}{$1$}      \\ \cline{2-4} 
\end{tabular}
\caption{To hendelser A og B i en \textit{Vierfeldertafel}}
\end{table}
\pause
\begin{itemize}[<+->]
\item Det er to spalter og to rekker, hver for en hendelse og motsetningen
\item I alle retninger kan man summe ting sammen
\item Det trenges bare tre utfylte felter for å regne ut resten
\end{itemize}
\end{frame}

\begin{frame}{\textit{Vierfeldertafel} (WANTED: norsk eller engelsk begrep)}
\begin{columns}
 \begin{column}{0.38\textwidth}
\begin{table}[h!]
\centering
\caption{Eksempel \textit{Vierfeldertafel}}
\label{tab:fourfold_eksempel1}
\begin{tabular}{l|ll|l}
\cline{2-3}
                            & $A$            & $A^C$            &                               \\ \hline
\multicolumn{1}{|l|}{$B$}   & 0.4   &     & \multicolumn{1}{l|}{0.533}   \\
\multicolumn{1}{|l|}{$B^C$} &   & 0.292 & \multicolumn{1}{l|}{} \\ \hline
                            &          &          & \multicolumn{1}{l|}{$1$}      \\ \cline{2-4} 
\end{tabular}
\end{table}

\begin{table}[h!]
\centering
%\caption{Eksempel \textit{Vierfeldertafel} ferdig utfylt}
\label{tab:fourfold_eksempel2}
\begin{tabular}{l|ll|l}
\cline{2-3}
                            & $A$            & $A^C$            &                               \\ \hline
\multicolumn{1}{|l|}{$B$}   & 0.4   & 0.133    & \multicolumn{1}{l|}{0.533}   \\
\multicolumn{1}{|l|}{$B^C$} & 0.175  & 0.292 & \multicolumn{1}{l|}{0.467} \\ \hline
                            & 0.575         & 0.425         & \multicolumn{1}{l|}{$1$}      \\ \cline{2-4} 
\end{tabular}
\end{table}
 \end{column}
 \pause
 \begin{column}{0.58\textwidth}
\begin{itemize}[<+->]
\item Det er gitt tre verdier, $P(B)$, $P(A\cap B)$, $P(A^C\cap B^C)$
\item Resten kan regnes ut ved summeformalen
\item Eksempel: $P(A^C\cap B)=P(A\cap B)-P(B)=0.533-0.4=0.133$
\item Kan brukes for å finne andre ting
\item Betinget sannsynlighet: $P(A|B)=\frac{P(A\cap B)}{P(B)}=\frac{0.4}{0.533}=0.75$
\end{itemize}
 \end{column}
\end{columns}
\end{frame}

\begin{frame}[fragile]{Sannsynlighetstre}
\begin{tikzcd}
           &                                                         & {} \arrow[ld, "P(A)" description] \arrow[rd, "P(\overline{A})" description] &                                                                               &                                   \\
           & {} \arrow[ld, "P(B|A)"'] \arrow[d, "P(\overline{B}|A)"] &                                                                             & {} \arrow[d, "P(B|\overline{A})"'] \arrow[rd, "P(\overline{B}|\overline{A})"] &                                   \\
P(A\cap B) & P(A \cap \overline{B})                                  &                                                                             & P(\overline{A} \cap B)                                                        & P(\overline{A} \cap \overline{B})
\end{tikzcd}
\end{frame}

\begin{frame}[fragile]{Sannsynlighetstre}
\begin{tikzcd}
    &                                                         & {} \arrow[ld, "0.575" description] \arrow[rd, "0.425" description] &                                                                               &       \\
    & {} \arrow[ld, "P(B|A)"'] \arrow[d, "P(\overline{B}|A)"] &                                                                    & {} \arrow[d, "P(B|\overline{A})"'] \arrow[rd, "P(\overline{B}|\overline{A})"] &       \\
0.4 & 0.175                                                   &                                                                    & 0.133                                                                         & 0.292
\end{tikzcd}
\end{frame}

\begin{frame}[fragile]{Sannsylighetstre}
\begin{tikzcd}
    &                                            & {} \arrow[ld, "0.575" description] \arrow[rd, "0.425" description] &                                            &       \\
    & {} \arrow[ld, "0.696"'] \arrow[d, "0.304"] &                                                                    & {} \arrow[d, "0.313"'] \arrow[rd, "0.687"] &       \\
0.4 & 0.175                                      &                                                                    & 0.133                                      & 0.292
\end{tikzcd}
\end{frame}



