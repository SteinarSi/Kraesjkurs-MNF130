\subsection{Korteste sti}
\begin{frame}[fragile]{Korteste sti}
    Vi kan bruke BFS til å finne den korteste stien i en graf, og dermed også den korteste ruten til Kvarteret. Men noen kanter/veier tar lengre tid å gå enn andre, så i praksis fungerer det ikke så bra.
    \begin{columns}
        \begin{column}{0.48\textwidth}
            \begin{tikzcd}
            \text{Høytek} \arrow[r] \arrow[d] & B \arrow[r] \arrow[d] & \text{Kvarteret}   \\
            A \arrow[r]                & C \arrow[r]           & D \arrow[u]
            \end{tikzcd}
        \end{column}
        \pause
        \begin{column}{0.48\textwidth}
            \begin{tikzcd}
            \text{Høytek} \arrow[r, "2"] \arrow[d, "1"] & B \arrow[r, "8"] \arrow[d, "3"] & \text{Kvarteret}        \\
            A \arrow[r, "5"]                     & C \arrow[r, "1"]                & D \arrow[u, "2"]
            \end{tikzcd}
        \end{column}
    \end{columns}
    \pause
    \begin{definition}[Vektet graf]
        En vektet graf er en der alle kantene også har en vekt/kostnad.
    \end{definition}
\end{frame}

\begin{frame}[fragile]{Dijkstras algoritme}
    Dijkstras algoritme fungerer utmerket for å finne den korteste stien i en vektet graf, så lenge vektene er positive tall.
    \begin{minted}[fontsize=\scriptsize]{python}
def dijkstra(start, end, graph):
    dist = { u: 99999999999999 for u  in graph }
    dist[start] = 0
    queue = [start]
    while len(queue) > 0:
        next = min(queue, key=lambda u: dist[u]) # finner noden med lavest distanse
        queue.remove(next)
        for (v, weight) in graph[next]:
            if dist[v] < dist[next] + weight:
                dist[v] = dist[next] + weight
                queue.append(v)
    return dist
    \end{minted}
\end{frame}