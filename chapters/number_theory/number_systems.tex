\subsection*{Tallsystem}
\begin{frame}
\begin{block}{Tallsystem}
En representasjon av tall med forskjellige tegn med en base
\end{block}
\pause

\medskip

\begin{table}[]
\centering
\label{tab:tallsystemer}
\begin{tabular}{l|l|l|l|l}
Navn & Sifre & 5 & 11 & 34 \\ \hline
Desimal (b=10) & 0-9 & 5& 11 & 34 \\
Binær (b=2) & 0-1 & 101 & 1011 & 100010 \\
Octal (b=8) & 0-7 & 5&  13 & 42\\
Hexadesimal (b=16) & 0-9,a-f& 5& B& 22\\
base=13 & 0-9,a-c & 5& B& 28
\end{tabular}
\caption{Eksempler på forskjellige tallsystemer}
\end{table}
\end{frame}

\subsection*{Konvertering av tallsystem}
\begin{frame}[fragile]{Desimal til base b}
\begin{columns}
    \begin{column}{0.30\textwidth}
    \begin{block}{pseudokode}
            tall n til base b:\\
            next digit = $n\%b$\\
            $n=\floor{\frac{n}{b}}$\\
            forsette med det til $n=0$
    \end{block}
    \end{column}
 	\pause
    \begin{column}{0.66\textwidth}
\begin{table}
\begin{tabular}{r|c|r}
n & nextDigit & output \\ \hline
22 & & 0 \\
11 & 0 & 0\\
5 & 1 & 10\\
2 & 1 & 110\\
1 & 0 & 0110\\
0 & 1 & 10110
\end{tabular}
\caption{Eksempel for dec til base 2}
\end{table}
 	\end{column}
 	\end{columns}
\end{frame}

\begin{frame}[fragile]{Base b til desimal}
\begin{columns}
    \begin{column}{0.30\textwidth}
\begin{block}{pseudokode}
    tall n og base b\\
    $sum=0$; $index=0$\\
    starter med først siffer $s$:\\
    $sum+=base^{index}\cdot s$\\
    $index+=1$\\
    forsette med hver siffer $s$
\end{block}
 	\end{column}
 	\pause
    \begin{column}{0.66\textwidth}
\begin{table}
%\begin{tabular}{r|r|r}
%toAdd & n & sum \\ \hline
% & 10110 & 0\\
%0 & 1011 & 0\\
%2 & 101 & 2\\
%4 & 10 & 6\\
%0 & 1 & 6\\
%16 & 0 & 22
%\end{tabular}
\begin{tabular}{c|r|r|r|r|r}
    tall & 1 & 0 & 1 & 1 & 0 \\ \hline
    base & 16 & 8 & 4 & 2 & 1 \\ \hline
    produkt & 16 & 0 & 4 & 2 & 0 
\end{tabular}
\caption{Eksempel for 2 til dec}
\end{table}
\begin{center}
$16+0+4+2+0=22$    
\end{center}

 	\end{column}
 	\end{columns}
\end{frame}