\subsection*{GCD, LCM}

\begin{frame}{GCD}
    La oss si at vi har en brøk: $\frac{12}{18}$. Hvordan forenkle den?\\
    \pause
    Vi deler telleren og nevneren på 6, og får $\frac{2}{3}$. Men hvorfor 6?\\
    \pause
    Fordi 6 er det største tallet som deler både $12$ og $18$. Den største felles faktoren er 6.

    \begin{block}{Største felles faktor (/Greatest Common Divisor)}
        $gcd(a,b) := $ det største tallet som deler både a og b\\
        Eksempel: $gcd(12, 18) = 6$.
    \end{block}

    \pause
    \begin{block}{Relativt primisk (/co-prime)}
        To heltall $a$ og $b$ kalles \emph{relativt prime}, eller \emph{co-prime}, dersom $gcd(a,b)=1$.\\
        Eksempel: $8$ og $21$ er relativt prime, siden $gcd(8,21) = 1$.
    \end{block}
\end{frame}

\begin{frame}{LCM}
    Vi har en sum av to brøker: $\frac{5}{18} + \frac{7}{12}$. Hvordan forenkle det?\\
    \pause
    Vi ganger den venstre med 2, og den høyre med 3, så begge får 36 i nevneren:\\
     $\frac{10}{36} + \frac{21}{36} = \frac{31}{36}$. Men hvorfor 36?\\
    \pause
    Fordi 36 er det laveste tallet som begge nevnerene kan ganges opp til, deres laveste felles multiplum er 36.
    \begin{block}{Laveste felles multiplum (/Least common multiple)}
        $lcm(a,b) :=$ det minste tallet som kan deles av både a og b\\
        Eksempel: $lcm(18,12)=36$\\
        Det er alltid sant at $a \cdot b = lcm(a,b) \cdot gcd(a,b)$, så vi kan regne ut $LCM$ med $lcm(a,b) = \frac{a \cdot b}{gcd(a,b)}$.
    \end{block}
\end{frame}

\subsection*{Euklids algoritme}
\begin{frame}[fragile]{Euklids algoritme}
\begin{columns}
    \begin{column}{0.45\textwidth}
\begin{python}
def gcd(a, b):
    while b != 0:
        r = a % b
        a = b
        b = r
    return a
\end{python}
 	\end{column}
 	\pause
    \begin{column}{0.45\textwidth}
        $gcd($\mathunderline{red}{1180},$\mathunderline{blue}{482}):$\\[2mm]
        \pause
        $\mathunderline{red}{1180} = 2 \cdot \mathunderline{blue}{482} + \mathunderline{green}{216}$\\
        \pause
        $\mathunderline{blue}{482} = 2 \cdot \mathunderline{green}{216} + \mathunderline{red}50$\\
        \pause
        $\mathunderline{green}{216} = 4 \cdot \mathunderline{red}{50} + \mathunderline{blue}{16}$\\
        \pause
        $\mathunderline{red}{50} = 3 \cdot \mathunderline{blue}{16} + \underline{2}$\\
        \pause
        $\mathunderline{blue}{16} = 8 \cdot \underline{2} + 0$\\[2mm]

        \pause
        $gcd(1180,482) = 2$.
 	\end{column}
\end{columns}


\end{frame}

\begin{frame}[fragile]{}
\begin{block}{Utvidet Euklids algoritme}
Regner ut to parameter $s$ og $t$ slik at $gcd(a,b)$ kan skrives som linærkombinasjon\\
$gcd(a,b)=s\cdot a+t\cdot b$\\
$gcd(12,28)=4=-2\cdot 12 + 1\cdot 28$\medskip

Kan brukes for å finne multiplikativt invers\\
Multiplikativ inverse finnes dersom $gcd(a,b)=1$
\end{block}
\pause

\begin{block}{Finne multiplikativt invers for $a$ med $mod\, m$}
\begin{itemize}
\item Funker bare dersom $gcd(a,m)=1$\\
\item Regn ut linærkombinasjon $gcd(a,b)=s\cdot a+t\cdot b$ med gcd
\item $a\cdot x \equiv 1 (mod\, m)$ er multiplicative inverse
\end{itemize}
\end{block}
\end{frame}

\begin{frame}{Extended Euklids algoritme}
    \begin{columns}
        \begin{column}{0.45\textwidth}
             gcd(26,7)\\
             
             $(26)=3\cdot (7)+(5)$\\
             $(7)=1\cdot (5)+(2)$\\
             $(5)=2\cdot(2)+(1)$\\
             $(2)=2\cdot(1)+(0)$\\
             
             gcd(26,7)=1
        \end{column}
        \pause
        \begin{column}{0.45\textwidth}
            Nå går vi tilbake:\\
            $(5)=2\cdot(2)+1$\\
            
            $\implies 1=(5)-2\cdot (2)$\\
            $=(5)-2\cdot ((7)-(5))$\\
            $=3\cdot (5)-2\cdot(7)$\\
            $=3\cdot ((26)-3\cdot (7))-2\cdot (7)$\\
            $=3\cdot (26)-11\cdot (7)$
        \end{column}
    \end{columns}
\end{frame}


\begin{frame}{Eksempel Multiplikativt Invers}
\begin{itemize}[<+->]
\item Hva er multiplikativt invers av $7\, mod\, 26$? ($a\cdot 7=1\, mod\, 26$)
\item $gcd(a,m) = gcd(7,26)=1$ $\rightarrow$ har multiplikativt invers
\item Linærkombinasjon fra gcd: $3\cdot (26)-11\cdot (7)=1$
\item $[1]_{26}=[3\cdot (26)-11\cdot (7)]_{26}=[-11\cdot 7]_{26}$
\item $a=-11$
\item $[-11]_{26}=[26-11]_{26}=[15]_{26}=15$
\item 15 er inverse av 7 modulo 26

\end{itemize}
\end{frame}