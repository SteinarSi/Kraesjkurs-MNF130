\subsection*{Primtall, Greatest common divisor, Least common multiple}
\begin{frame}
\begin{block}{Primtall}
Et tall som bare kan deles av seg selv og 1\\
Eksempler: 2,3,5,7,11,13,...
\end{block}
\pause
\begin{block}{Største felles faktor (/Greatest common divisor)}
$gcd(a,b) := $det største tallet som deler både a og b\\
Eksempel: $gcd(4,6)=2$\\
Co-prime: a og b er co-prime dersom $gcd(a,b)=1$
\end{block}
\pause
\begin{block}{Laveste felles multiplum (/Least common multiple)}
$lcm(a,b) := $ det minste tallet som kan deles av både a og b\\
Eksempel: $lcm(4,6)=12$\\
hvis $a\cdot b=gcd(a,b)\cdot lcm(a,b)$\\
$gcd(a,b) = 1$ $\rightarrow lcm(a,b) = a\cdot b$
\end{block}
\end{frame}

\subsection*{Euklids algoritme}
\begin{frame}[fragile]{Euklids algoritme}
\begin{columns}
    \begin{column}{0.45\textwidth}
\begin{minted}[fontsize=\scriptsize]{python}
def gcd(a, b):
    while b != 0:
        r = a % b
        a = b
        b = r
    return a
\end{minted}
 	\end{column}
 	\pause
    \begin{column}{0.45\textwidth}
    \begin{center}
       gcd(28,12)\\
       
    $28=2\cdot 12+4$\\
    $12=3\cdot 4+0$\\
    
    gcd(28,12)=4 
    \end{center}
 	\end{column}
\end{columns}


\end{frame}

\begin{frame}[fragile]{}
\begin{block}{Utvidet Euklids algoritme}
Regner ut to parameter $s$ og $t$ slik at $gcd(a,b)$ kan skrives som linærkombinasjon\\
$gcd(a,b)=s\cdot a+t\cdot b$\\
$gcd(12,28)=4=-2\cdot 12 + 1\cdot 28$\medskip

Kan brukes for å finne multiplikativt invers\\
Multiplikativ inverse finnes dersom $gcd(a,b)=1$
\end{block}
\pause

\begin{block}{Finne multiplikativt invers for $a$ med $mod\, m$}
\begin{itemize}
\item Funker bare dersom $gcd(a,m)=1$\\
\item Regn ut linærkombinasjon $gcd(a,b)=s\cdot a+t\cdot b$ med gcd
\item $a\cdot x \equiv 1 (mod\, m)$ er multiplicative inverse
\end{itemize}
\end{block}
\end{frame}

\begin{frame}{Extended Euklids algoritme}
    \begin{columns}
        \begin{column}{0.45\textwidth}
             gcd(26,7)\\
             
             $(26)=3\cdot (7)+(5)$\\
             $(7)=1\cdot (5)+(2)$\\
             $(5)=2\cdot(2)+(1)$\\
             $(2)=2\cdot(1)+(0)$\\
             
             gcd(26,7)=1
        \end{column}
        \pause
        \begin{column}{0.45\textwidth}
            Nå går vi tilbake:\\
            $(5)=2\cdot(2)+1$\\
            
            $\implies 1=(5)-2\cdot (2)$\\
            $=(5)-2\cdot ((7)-(5))$\\
            $=3\cdot (5)-2\cdot(7)$\\
            $=3\cdot ((26)-3\cdot (7))-2\cdot (7)$\\
            $=3\cdot (26)-11\cdot (7)$
        \end{column}
    \end{columns}
\end{frame}


\begin{frame}{Eksempel Multiplikativt Invers}
\begin{itemize}[<+->]
\item Hva er multiplikativt invers av $7\, mod\, 26$? ($a\cdot 7=1\, mod\, 26$)
\item $gcd(a,m) = gcd(7,26)=1$ $\rightarrow$ har multiplikativt invers
\item Linærkombinasjon fra gcd: $3\cdot (26)-11\cdot (7)=1$
\item $[1]_{26}=[3\cdot (26)-11\cdot (7)]_{26}=[-11\cdot 7]_{26}$
\item $a=-11$
\item $[-11]_{26}=[26-11]_{26}=[15]_{26}=15$
\item 15 er inverse av 7 modulo 26

\end{itemize}
\end{frame}