\subsection{Proposisjoner}
\begin{frame}{Proposisjoner}
\begin{itemize}
    \item En proposisjon er et uttrykk med en sannhetsverdi, som alltid er enten Sann (T) elller Usann (F).
    \item Ofte setter vi dem i variabler, for å gjøre det lettere å lese.
\end{itemize}
\pause
\begin{block}{Proposisjoner}
    p := 2 < 3 \\
    q := ''Månen er laget av ost''\\
    r := ''MNF130 er et nyttig fag for informatikere''
\end{block}
\pause
\begin{block}{Ikke proposisjoner}
    ''Hva skal vi ha til middag i dag?''\\
    x + y < z
\end{block}
\end{frame}

\begin{frame}{$\lnot$, og sannhetstabeller}
\begin{itemize}
    \item Gitt en proposisjon $p$, kan vi representere den motsatte proposisjonen som $\lnot p$.
    \item Vi kan tegne opp alle mulige verdier et slikt uttrykk kan ha i en tabell.
\end{itemize}
\pause

\begin{block}{Eksempler}
    La $p := 2 < 3$. \\
    $p$ = ''Det er sant at 2 er mindre enn 3'' \\
    $\lnot p$ = ''Det er ikke sant at 2 er mindre enn 3''
\end{block}
\pause
\begin{tabular}{c|c}
$p$ & $\lnot p$ \\ \hline
T & F \\
F & T \\ \hline
\end{tabular}
\end{frame}

\begin{frame}{$\lor$ og $\land$}
    \begin{itemize}
        \item Vi kan slå sammen flere proposisjoner til et større uttrykk på mange forskjellige måter.
    \end{itemize}
    \begin{block}{Eksempel}
        La $p$ := Jorden er flat, og $q$ := Månen er flat\\
        $p \lor q$ = Jorden er flat \textbf{eller} månen er flat\\
        $p \land q$ = Jorden er flat \textbf{og} månen er flat\\
    \end{block}
    \pause
    \begin{tabular}{c|c|c|c}
        $p$ & $q$ & $p \lor q$ & $p \land q$ \\ \hline
        T & T & T & T \\
        T & F & T & F \\
        F & T & T & F \\
        F & F & F & F \\
    \end{tabular}
\end{frame}

\begin{frame}{$\rightarrow$ og $\leftrightarrow$}
    \begin{itemize}
        \item En proposisjon kan implisere en annen proposisjon.
    \end{itemize}
    \begin{block}{Eksempel}
        La $p$ := ''Det regner ute'', og $q$ := ''Bakken er våt''\\
        $p \rightarrow q$ = ''Hvis det regner ute, blir bakken våt''\\
        $p \leftrightarrow q$ = ''Det regner ute hvis og bare hvis bakken er våt''
    \end{block}
    \pause
    \begin{tabular}{c|c|c|c}
        $p$ & $q$ & $p \rightarrow q$ & $p \leftrightarrow q$ \\ \hline
        T & T & T & T \\
        T & F & F & F \\
        F & T & T & F \\
        F & F & T & T \\
    \end{tabular}
\end{frame}

\begin{frame}{Egenskaper}
    \begin{itemize}
        \item Når to logiske uttrykk alltid har samme verdi, sier vi at de er \emph{logisk ekvivalente}.
        \pause
        \item Om et uttrykk alltid er sant, uavhengig av innholdet, kaller vi det en \emph{tautologi}.
        \pause
        \item Et uttrykk er \emph{tilfredsstilbart} om det finnes en kombinasjon av variabler slik at uttrykket blir T.
    \end{itemize}
    \pause
    \begin{tabular}{c|c|c|c|c}
         $p$ & $\lnot p$ & $\lnot \lnot p$ & $p \lor \lnot p$ & $p \land \lnot p$\\ \hline
         T & F & T & T & F\\
         F & T & F & T & F
    \end{tabular}
    \pause
    \begin{itemize}
        \item Konklusjon 1: $p \equiv \lnot \lnot p$, og er logisk ekvivalente
        \pause
        \item Konklusjon 2: $p \lor \lnot p \equiv T$, og er en tautologi
        \pause
        \item Konklusjon 3: $p \land \lnot p \equiv F$, og er ikke tilfredsstilbart.
    \end{itemize}
    \pause
    De fleste oppgavene om dette kan løses bare ved å tegne opp sannhetstabellen:
    \begin{itemize}
        \item To uttrykk har like kolonner $\rightarrow$ uttrykkene er logisk ekvivalente
        \pause
        \item Et uttrykk har kun T-er $\rightarrow$ uttrykket er en tautologi
        \pause
        \item Et uttrykk har minst én T $\rightarrow$ uttrykket er tilfredsstilbart
    \end{itemize}
\end{frame}

\begin{frame}{Noen viktige logiske ekvivalenser}
    \begin{columns}
    \begin{column}{0.32\textwidth}
        \begin{tabular}{l|c}
        Ekvivalens & Navn \\ \hline
        $p \land T \equiv p$ & Identity\\
        $p \lor F \equiv p$ \\ \hline
        
        $p \lor T \equiv T$ & Domination\\
        $p \land F \equiv F$\\ \hline
        
        $p \lor p \equiv p$ & Idempotent\\
        $p \land p \equiv p$ \\ \hline
        
        $p \equiv \lnot \lnot p$ & Negation\\ \hline
        
        $p \lor q \equiv q \lor p$ & Commutative\\
        $p \land q \equiv q \land p$ \\
        

    \end{tabular}
    \end{column}
    \begin{column}{0.52\textwidth}
        \begin{tabular}{l|c}
        Ekvivalens & Navn \\ \hline
        
        $(p \lor q) \lor r \equiv p \lor (q \lor r)$ & Associative\\
        $(p \land q) \land r \equiv p \land (q \land r)$ \\ \hline
        
        $p \lor (q \land r) \equiv (p \lor q) \land (p \lor r)$ & Distributive\\
        $p \land (q \lor r) \equiv (p \land q) \lor (p \land r)$ \\ \hline
        
        $\lnot (p \land q) \equiv \lnot p \lor \lnot q$ & De Morgan \\
        $\lnot (p \lor q) \equiv \lnot p \land \lnot q$ \\ \hline
        
        $p \lor (p \land q) \equiv p$ & Absorption \\
        $p \land (p \lor q) \equiv q$ \\ \hline
        
        $p \lor \lnot p \equiv T$ & Negation \\
        $p \land \lnot p \equiv F$ \\
        \end{tabular}
    \end{column}
\end{columns}
\end{frame}

\begin{frame}{Flere viktige logiske ekvivalenser}
    \begin{columns}
    \begin{column}{0.48\textwidth}
        \begin{tabular}{c}
            Ekvivalenser med $\rightarrow$ \\ \hline
            $p \rightarrow q \equiv \lnot p \lor q$ \\
            $p \rightarrow q \equiv \lnot p \rightarrow \lnot q$ \\
            $p \lor q \equiv \lnot p \rightarrow q$ \\
            $p \land q \equiv \lnot (p \rightarrow \lnot q)$ \\
            $\lnot (p \rightarrow q) \equiv p \land \lnot q$ \\
            $(p \rightarrow q) \land (p \rightarrow r) \equiv p \rightarrow (q \land r)$ \\
            $(p \rightarrow q) \land (q \rightarrow r) \equiv (p \lor q) \rightarrow r$ \\
            $(p \rightarrow q) \lor (p \rightarrow r) \equiv p \rightarrow (q \lor v)$ \\
            $(p \rightarrow r) \lor (q \rightarrow r) \equiv (p \land q) \rightarrow r$
        \end{tabular}
    \end{column}
    \begin{column}{0.48\textwidth}
        \begin{tabular}{c}
            Ekvivalenser med $\leftrightarrow$ \\ \hline
            $p \leftrightarrow q \equiv p \rightarrow q \land q \rightarrow p$ \\
            $p \leftrightarrow q \equiv \lnot p \leftrightarrow \lnot q$ \\
            $p \leftrightarrow q \equiv (p \land q) \lor (\lnot p \land \lnot q)$\\
            $\lnot (p \leftrightarrow q) \equiv p \rightarrow \lnot q$
        \end{tabular}
    \end{column}
    \end{columns}
\end{frame}

\begin{frame}{Typisk logikkoppgave}
    \begin{columns}
    \begin{column}{0.48\textwidth}
        Vis at $(p \land \lnot q) \rightarrow \lnot r$ og $(p \land r) \rightarrow q$ er logisk ekvivalente, ved å bruke enkle logiske ekvivalenser.\\[1cm]
        % Egenskapene vi trenger:
        % \begin{itemize}
        %     \item $a \rightarrow b \equiv \lnot a \lor b$
        %     \item $\lnot (a \land b) \equiv \lnot a \lor \lnot b$
        %     \item $\lnot (\lnot a) \equiv a$
        % \end{itemize}
    \end{column}
    \begin{column}{0.48\textwidth}
            \pause
            $(p \land \lnot q) \rightarrow \lnot r$ \\
            \pause
            $\equiv \lnot (p \land \lnot q) \lor \lnot r$ \\
            \pause
            $\equiv (\lnot p \lor \lnot \lnot q) \lor \lnot r$ \\
            \pause
            $\equiv (\lnot p \lor q) \lor \lnot r$ \\
            \pause
            $\equiv \lnot p \lor q \lor \lnot r$ \\
            \pause
            $\equiv \lnot p \lor \lnot r \lor q$ \\
            \pause
            $\equiv \lnot (p \land r) \lor q$ \\
            \pause
            $\equiv (p \land r) \rightarrow q$
            $\qed$
    \end{column}
    \end{columns}
\end{frame}

\begin{frame}{Predikater}
    \begin{itemize}
        \item Et predikat er bare en funksjon som returnerer T eller F.
        \item Gitt et predikat og riktig antall argumenter, kan vi evaluere det som en vanlig proposisjon.
    \end{itemize}
    \begin{block}{Eksempler på predikater}
        $P(x, y, z) = x + y < z$\\
        $Q(s)$ = $s$ contains $'a'$
    \end{block}
    \pause
    \begin{block}{Eksempler på evaluering}
        $P(1, 2, 3) = 1 + 2 < 3 = 3 < 3 = F$ \\ 
        $Q($\enquote{Steinar}$) = $\enquote{Steinar} contains $'a' = T$
    \end{block}

\end{frame}