\subsection{Induksjon}
\begin{frame}{Matematisk induksjon}
    \begin{columns}
        \begin{column}{0.65\textwidth}
            Et induksjonsbevis for en påstand $P(x)$ består av to steg:
            \begin{itemize}
                \item Vis at $P(0)$ stemmer
                \item Vis at om $P(k)$ stemmer, må $P(k+1)$ stemme
            \end{itemize}
        
            \pause
            Siden $P(0)$ stemmer, må $P(1)$ stemme, og\\
            Siden $P(1)$ stemmer, må $P(2)$ stemme, og\\
            Siden $P(2)$ stemmer, må $P(3)$ stemme, og\\
            ...\\
            Siden $P(k)$ stemmer, må $P(k+1)$ stemme.\\[1.5mm]
        
            \pause
            Dermed har vi bevist at $\forall x : x \geq 0 \rightarrow P(x)$.
        \end{column}
        \pause
        \begin{column}{0.25\textwidth}
            \begin{figure}
                \includegraphics[scale=0.1]{images/domino.PNG}
            \end{figure}
        \end{column}
    \end{columns}
\end{frame}

\begin{frame}{Oppgave: vis at $\sum_{i=0}^{n} i = 0 + 1 + 2 + ... + n = \frac{n(n+1)}{2}$.}
    \pause    
    Vi lar $P(k) := \sum_{i=0}^{k} i = \frac{k(k+1)}{2}$\\

    \pause
    Vi begynner med base case, at påstanden holder for $P(0)$:\\
    $P(0): \sum_{i=0}^{0} i = 0$ \pause $ = \frac{0(0+1)}{2}$ \checkmark\\
    
    \pause
    Induksjonshypotesen vår er at påstanden holder for en vilkårlig $k$: \\
    $P(k): \sum_{i=0}^{k} i = \frac{k(k+1)}{2}$\\
    \pause
    Vi må vise at det impliserer at påstanden også holder for $P(k+1)$:\\
    $P(k+1): \sum_{i=0}^{k+1} i = \frac{(k+1)((k+1)+1)}{2}$ \pause $ = \frac{(k+1)(k+2)}{2}$\\

    \pause
    $\sum_{i=0}^{k+1} i = 0 + 1 + 2 + ... + k + (k+1)$ \pause $= (\sum_{i=0}^{k} i )+ (k+1)$\\
    \pause
    $=\frac{k(k+1)}{2} + k + 1$\\
    \pause
    $= \frac{k(k+1)}{2}+\frac{2(k+1)}{2} = \frac{k(k+1)+2(k+1)}{2}$\\
    \pause
    $= \frac{(k+1)(k+2)}{2}$ \pause \checkmark \\[2mm]
    \pause
    Siden påstanden holder for $P(0)$, og siden $P(k) \rightarrow P(k+1)$, vil påstanden holde for alle $n \geq 0$. $\qed$
\end{frame}

\begin{frame}{Sterk induksjon}
    Dette er veldig likt. Men istedet for å anta $P(k)$, antar man $P(0) \land P(1) \land .... \land P(k)$.\\
    Du kan tenke helt likt på disse to formene for induksjon, eneste forskjellen er at du kan anta litt mer i induksjonssteget.\\

    Hvis du allerede forstår induksjon, trenger du ikke definisjon for sterk induksjon. Du kan gjørde det med intuisjon.
\end{frame}

\subsection{Rekursjon}
\begin{frame}{Rekursive funksjoner}
    En rekursiv funksjon er en funksjon som refererer til seg selv.
    \pause
    \begin{block}{fakultet}
        $0! := 1$ \\
        $n! := n \cdot (n-1)!$
    \end{block}    
    \pause 
    $3! = 3 \cdot ~ 2!$\\
    $= 3 \cdot 2 \cdot 1!$\\
    $= 3 \cdot 2 \cdot 1 \cdot 0!$\\
    $= 3 \cdot 2 \cdot 1 \cdot 1$\\
    $= 6$
\end{frame}

\begin{frame}{Evaluering av en rekursiv funksjon}
    $fib(0) := 0$\\
    $fib(1) := 1$\\
    $fib(n) := fib(n-1) + fib(n-2)$\\
    
    \pause
    $fib(5) = fib(3) + fib(4)$\\
    $ = [fib(1) + fib(2)] + [fib(3) + fib(2)]$\\
    $ = [1 + fib(0) + fib(1)] + [fib(2) + fib(1) + fib(1) + fib(0)]$\\
    $ = [1 + 0 + 1] + [fib(1) + fib(0) + 1 + 1 + 0]$\\
    $ = [1 + 0 + 1] + [1 + 0 + 1 + 1 + 0]$\\
    $ = 5$
\end{frame}

\begin{frame}{Rekursive strukturer}
    Som med funksjoner kan vi også definere andre strukturer rekursivt.\\
    
    De defineres med et basissteg, altså et utgangspunkt, og et rekursivt steg for å utvide det.\\
    
    \pause
    \begin{block}{Eksempel}
        Basissteg: $7 \in \mathbb{S}$.\\
        Rekursivt steg: $a \in \mathbb{S} \rightarrow 10a \in \mathbb{S}$\\
        $\mathbb{S} = \{7, 70, 700, 7000, 70000, ....\}$
    \end{block}
\end{frame}

\subsection{Strukturell induksjon}
\begin{frame}{Strukturell induksjon}
    Ofte vil vi bevise egenskaper for rekursive strukturer. Det gjøres i to steg:
    \begin{itemize}
        \item Basissteg: vis at strukturen i utgangspunktet har egenskapen.
        \item Rekursivt steg: vis at om en rekursiv struktur allerede har en egenskap, vil en runde med rekursjon opprettholde den egenskapen.
    \end{itemize}
    
    \pause
    \begin{block}{}
        Basissteg: $7 \in \mathbb{S}$.\\
        Rekursivt steg: $a \in \mathbb{S} \rightarrow 10a \in \mathbb{S}$\\
        $\mathbb{S} = \{7, 70, 700, 7000, 70000, ....\}$
    \end{block}
    \begin{block}{Vis at alle tall i $\mathbb{S}$ er delelige på 7.}
        \pause
        Basisteg: $\mathbb{S}$ inneholder kun $7$, og $7$ mod $7$ = 0. \checkmark\\
        \pause
        Rekursivt steg: Vi antar at alle tall $a \in \mathbb{S}$ er delelig på 7. Så $a = 7k$, $k \in \mathbb{N}$.\\
        \pause
        I neste rekursjon legger vi til $10a$ i $\mathbb{S}$.\\
        \pause
        $10a=10 \cdot 7k = 7 \cdot (10ka)$, og er også delelig på 7.\\
        \pause
        En runde rekursjon opprettholder egenskapen. \checkmark\\
        \pause
        Derfor er alle tall i $\mathbb{S}$ delelige på 7. \qed
    \end{block}
\end{frame}

\begin{frame}{Større eksempel på strukturell induksjon}
    Basissteg: $(0, 0) \in \mathbb{T}$.\\
    Rekursivt steg: $(a, b) \in \mathbb{T} \rightarrow (a+1, b), (a+1, b+1) \in \mathbb{T}$.\\
    
    Oppgave: vis at $\forall a, b \in \mathbb{T} : a \geq b$.\\
    
    \pause
    Basissteg: $(0, 0) \in \mathbb{T}$, og $0 \geq 0$. \checkmark\\
    \pause

    Rerkursivt steg: vi antar at $\forall a, b \in \mathbb{T} : a \geq b$.\\
    \pause
    Hvert rekursive kall legger til to nye par, $(a+1, b)$ og $(a+1, b+1)$.\\
    \pause
    Vi ser på dem hver for seg:
    \begin{itemize}
        \item Om $a \geq b$ er $a+1 \geq b$.
        \item Om $a \geq b$ er $a+1 \geq b+1$. \checkmark
    \end{itemize}
    \pause
    Dermed kan vi konkludere med at $\forall a, b \in \mathbb{T} : a \geq b$. \qed
\end{frame}

\begin{frame}[fragile]{Strukturell induksjon i praksis}
    \begin{lstlisting}[language=Java]
public class SimpleList<T> {
    private final T[] content;
    private int size;

    public SimpleList(int capacity) {
        content = (T[]) new Object[capacity];
        size = 0;
    }
    public void add(T element) {
        content[size] = element;
        size++;
    }
    public T get(int index) { 
        return content[index]; 
    }
    public int size() { 
        return size; 
    }
}
    \end{lstlisting}

    \pause
    Spørsmål: er \pyth{.size()} korrekt?
\end{frame}
