\subsection{Induksjon}
\begin{frame}{Matematisk induksjon}
    Et induksjonsbevis for en påstand $P(x)$ består av to steg:\\
    \begin{itemize}
        \item Vis at $P(0)$ stemmer
        \item Vis at om $P(n)$ stemmer, må $P(n+1)$ stemme
    \end{itemize}
    Dermed har vi bevist at $\forall x : [ x \geq 0 \rightarrow P(x)]$
    
    \begin{figure}
        \centering
        \includegraphics[scale=0.1]{images/domino.PNG}
        % \caption{}
        \label{fig:my_label1}
    \end{figure}
\end{frame}

\begin{frame}{Apen og Stigen} %bild?
    Apen står på trinn 0.\\
    
    Hvis apen viser deg at den kan klatre opp til trinn 1, vet du om den kan klatre opp til trinn k?\\
\pause
    Nei, du vet bare at den kan klatre opp til trinn 1.\\
    
\pause
    Hvis apen viser deg at den kan klatre opp fra et vilkårlig trinn til det neste, vet du nå om den kan klatre opp til trinn k?\\
\pause
    Ja, vi vet at den allerede står på trinn 0 og den kan klatre fra er trinn til neste. Den kan derfor klatre opp hvert trinn.
    
\end{frame}
\begin{frame}{Eksempel på matematisk induksjon}
    Vis at $\sum_{i=0}^{n} i = \frac{n(n+1)}{2}$.\\
\pause    
    Vi lar $P(n) := \sum_{i=0}^{n} i = \frac{n(n+1)}{2}$\\%kan jeg aendre fra x til n?
    
\pause 
    Vis at $P(0)$ stemmer\\
\pause
    Vi begynner med base case, at påstanden holder for $P(0)$.\\
    $P(0): \sum_{i=0}^{0} i = \frac{0(0+1)}{2} = 0$ \checkmark\\
\end{frame}

\begin{frame}{Eksempel på matematisk induksjon (fortsettelse)}
    Vis at om $P(n)$ stemmer, må $P(n+1)$ stemme\\
    Nå antar vi at påstanden holder for $n$: \\
    $P(n): \sum_{i=0}^{n} i = \frac{n(n+1)}{2}$\\
    Nå viser vi at det medfører at påstanden også må hold for $n+1$:\\
    $P(n+1): \sum_{i=0}^{n+1} i = \frac{(n+1)((n+1)+1)}{2} = \frac{(n+1)(n+2)}{2}$\\

    $\sum_{i=0}^{n+1} i =1 + 2 + ... + n + (n+1) = (\sum_{i=0}^{n} i )+ (n+1)$\\
\pause
    med $P(n)$ $\implies \frac{n(n+1)}{2} + n + 1$\\
    $ = \frac{n(n+1)}{2}+\frac{2(n+1)}{2} = \frac{n(n+1)+2(n+1)}{2}$\\
    $ = \frac{(n+1)(n+2)}{2}$ \checkmark
\end{frame}

\begin{frame}{Sterk induksjon}
    Dette er veldig likt. Men istedet for å anta $P(k)$, antar man $P(0) \land P(1) \land .... \land P(k)$.\\
    Du kan tenke helt likt på disse to formene for induksjon, eneste forskjellen er at du kan anta litt mer i induksjonssteget.\\

    Hvis du allerede forstår induksjon, trenger du ikke definisjon for sterk induksjon. Du kan gjørde det med intuisjon.
\end{frame}

\begin{frame}{Rekursive funksjoner}
    En rekursiv funksjon er en funksjon som refererer til seg selv.
    \pause
    \begin{block}{fakultet}
        0! := 1 \\
        n! := n * (n-1)!\\
    \end{block}    
    \pause 
    3! = 3 * 2!\\
     = 3 * 2 * 1!\\
     = 3 * 2 * 1 * 0!\\
     = 3 * 2 * 1 * 1\\
     = 6
\end{frame}

\begin{frame}{Evaluering av en rekursiv funksjon}
    $fib(0) := 0$\\
    $fib(1) := 1$\\
    $fib(n) := fib(n-1) + fib(n-2)$\\
    
    \pause
    $fib(5) = fib(3) + fib(4)$\\
    $ = [fib(1) + fib(2)] + [fib(3) + fib(2)]$\\
    $ = [1 + fib(0) + fib(1)] + [fib(2) + fib(1) + fib(1) + fib(0)]$\\
    $ = [1 + 0 + 1] + [fib(1) + fib(0) + 1 + 1 + 0]$\\
    $ = [1 + 0 + 1] + [1 + 0 + 1 + 1 + 0]$\\
    $ = 5$
\end{frame}

\begin{frame}{Rekursive definisjoner}
    Som med funksjoner kan vi også definere andre strukturer rekursivt. Med 'andre strukturer' mener vi sett 90\% av tiden.\\
    
    De defineres med et basissteg, dvs utgangspunktet, og et rekursivt steg for å utvide det.\\
    
    \pause
    \begin{block}{Eksempel}
        Basissteg: $7 \in \mathbb{S}$.\\
        Rekursivt steg: $a \in \mathbb{S} \rightarrow 10a \in \mathbb{S}$\\
        $\mathbb{S} = \{7, 70, 700, 7000, 70000, ....\}$
    \end{block}
\end{frame}

\begin{frame}{Strukturell induksjon}
    Ofte vil vi bevise egenskaper for rekursive strukturer.\\
    Det gjøres ved to steg:\\
    \begin{itemize}
        \item Basissteget: vis at egenskap holder for strukturens basissteg.
        \item Det rekursive steget: vis at om en egenskap allerede holder for en struktur, vil en runde med rekursjon opprettholde den egenskapen.
    \end{itemize}
    
    \pause
    \begin{block}{Vis at alle tall i $\mathbb{S}$ er delelige på 7.}
        Basisteget: $\mathbb{S}$ inneholder bare $7$, og $7$ mod $7$ = 0. \checkmark\\
        Rekursive steget: Vi antar at et tall $a \in \mathbb{S}$ er delelige på 7. Så $a = 7k$, $k \in \mathbb{N}$\\
        Vi vet at også $10a \in \mathbb{S}$.\\
        $10a=10 * 7k$ så alle tall i $\mathbb{S}$ er delelige på 7.\checkmark
    \end{block}
\end{frame}

\begin{frame}{Større eksempel på strukturell induksjon}
    Basissteg: $(0, 0) \in \mathbb{T}$\\
    Rekursivt steg: $(a, b) \in \mathbb{T} \rightarrow (a+1, b) \in \mathbb{T} \land (a+1, b+1) \in \mathbb{T}$.\\
    
    Oppgave: vis at $\forall a, b : [(a, b) \in \mathbb{T} \rightarrow a \geq b]$.\\
    
    \pause
    Basissteg: $(0, 0) \in \mathbb{T}$, og $0 \geq 0$. \checkmark\\
    Rerkursivt steg: vi antar at $\forall a, b : [(a, b) \in \mathbb{T} \rightarrow a \geq b]$. Hvert rekursive kall legger til to nye par, $(a+1, b)$ og $(a+1, b+1)$. Vi ser på dem hver for seg:
    \begin{itemize}
        \item Om $a \geq b$ er $a+1 \geq b$.
        \item Om $a \geq b$ er $a+1 \geq b+1$. \checkmark
    \end{itemize}
    Dermed kan vi konkludere med at $\forall a, b : [(a, b) \in \mathbb{T} \rightarrow a \geq b]$.
\end{frame}
