\subsection*{Følger og summer}

\begin{frame}{Følger (Sequences)}
    \pause
    \begin{block}{Definisjon}
        \begin{itemize}
            \item diskret struktur som representerer en ordnet list\\
            \item funksjon fra en undermengde av heltall (vanligvis $\mathbb{N}$ eller $\mathbb{N}_0$) til en mengde $S$\\
            \item vanlig notasjon er $\{a_n\}$ der $a_n$ kalles for en term av følgen. IKKE bland med mengder!\\
            \item Eksempel: Følge $\{a_n\}$ der $a_n = \frac{1}{n}$, altså $a_1 = \frac{1}{1} = 1, a_2 = \frac{1}{2}, ...$
        \end{itemize}
    \end{block}
\end{frame}

\begin{frame}{Geometrisk Progresjon (Geometric Progression)}
    \pause
    \begin{block}{Definisjon}
        \begin{itemize}
            \item Følge som har form $a, ar, ar^2, ..., ar^n, ...$
            \item $a$ kalles for \textit{startterm} (initial term) og $r$ kalles for \textit{fellesforhold} (common ratio)
        \end{itemize}
    \end{block}
    \pause
    Eksempler:
    \begin{itemize}
        \item Følge $\{b_n\}$ der $b_n = 2 \cdot 5^n$, altså $b_1 = 2 \cdot 5^0 = 2, b_2 = 10, b_3 = 50, ...$
        \item Følge $\{c_n\}$ der $c_n = 6 \cdot \big\( \frac{1}{3} \big\)^n$, altså $c_1 = 6 \cdot \big\( \frac{1}{3} \big\)^0 = 6, c_2 = 2, c_3 = \frac{2}{3}, ...$
    \end{itemize}
\end{frame}

\begin{frame}{Aritmetisk Progresjon (Arithmetic Progression)}
    \pause
    \begin{block}{Definisjon}
        \begin{itemize}
            \item følge som har form $a, a + d, a + 2d, ..., a + nd, ...$
            \item $a$ kalles for \textit{startterm} og $d$ kalles for \textit{fellesdifferanse} (common difference)
        \end{itemize}
    \end{block}
    \pause
    Eksempler:
    \begin{itemize}
        \item Følge $\{s_n\}$ der $s_n = -1 + 4n$, altså $s_1 = -1 + 4 \cdot 0 = -1, s_2 = 3, s_3 = 7, ...$
        \item Følge $\{t_n\}$ der $t_n = 7 - 3n$, altså $t_1 = 7 - 3 \cdot 0 = 7, t_2 = 4, t_3 = 1, ...$
    \end{itemize}
\end{frame}

\begin{frame}{Rekurrensrelasjoner (Recurrence Relations)}
    \begin{block}{Definisjon}
        \begin{itemize}
            \item uttrykker $a_n$ med forrige termer i følgen, dvs. noen av $a_0, a_1, ..., a_{n-1}$
            \item en følge kalles for løsning av en rekurrensrelasjon hvis det tilfredstiller alle kravene.
        \end{itemize}
    \end{block}
    \pause
    Eksempler:
    \begin{itemize}
        \item La $\{a_n\}$
        \item $[8+21]_6 = [[8]_6 + [21]_6]_6 = [2 + 3]_6 = [5]_6 = 5$
\item Multiplikasjon: $[a\cdot b]_m = [[a]_m \cdot [b]_m]_m$
\item $[8 \cdot 21]_6 = [[8]_6 \cdot [21]_6]_6 = [2 \cdot 3]_6 = [6]_6 = 0$
\end{itemize}
\end{frame}

\begin{frame}{}
\begin{exampleblock}{Eksempel}
\begin{itemize}
\item $x \equiv 3\,(mod\, 5)$ eller $[x]_5=3$
\item $y \equiv 4\,(mod\, 5)$ eller $[y]_5=4$
\item Finn løsningen: $(3\cdot x+2\cdot y^2)\, mod\,5$
\end{itemize}
\end{exampleblock}
\pause
\medskip

$[3\cdot x+2\cdot y^2]_5=[[3\cdot x]_5+[2\cdot y^2]_5]_5$\\

$[3\cdot x]_5=[[3]_5\cdot [x]_5]_5=[3\cdot 3]_5=[9]_5=4$\\
$[2\cdot y^2]_5=[[2]_5\cdot [y\cdot y]_5]_5=[[2]_5\cdot [y]_5\cdot[y]_5]_5=[2\cdot 4\cdot 4]_5=[32]_5=2$\\

$[[3\cdot x]_5+[2\cdot y^2]_5]_5=[4+2]_5=[6]_5=1$
\end{frame}

\begin{frame}[fragile]{Modulo ved subtraksjon}
       Vi vet at vi har addisjon, men hva er med subtrasjon?\\

Substraksjon:\\
$[6-3]_8=[3]_8$\\
$[3-6]_8$?\\
\pause
$[3-6]_8=[-3]_8=[0-3]_8=[8-3]_8=[5]_8=5$\\

Subtraksjon fungerer også for modulo.  
\end{frame}

\begin{frame}{Modulo ved divisjon}
Vi vet at vi har multiplikasjon, men hva med divisjon?\\

Divisjon:\\
$[6/3]_8=[2]_8$? \pause ja, fordi $[2\cdot 3]_8=[6]_8$\\
$[3/6]_8$?\\
\pause
Nei, noen ganger fungerer det, noen ganger fungerer det ikke.\\

Vi kan ikke alltid dele!
\end{frame}