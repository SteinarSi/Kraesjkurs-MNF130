\subsection*{Følger og summer}

\begin{frame}{Følger (Sequences)}
    \pause
    \begin{block}{Definisjon}
        \begin{itemize}
            \item diskret struktur som representerer en ordnet liste\\
            \item funksjon fra en undermengde av heltall (vanligvis $\mathbb{N}$ eller $\mathbb{N}_0$) til en mengde $S$\\
            \item vanlig notasjon er $\{a_n\}$ der $a_n$ kalles for en term av følgen.
            \item OBS! IKKE bland med mengdenotasjon!
        \end{itemize}
    \end{block}
    \pause
    Eksempel: Følge $\{a_n\}$ der $a_n = \frac{1}{n}$, altså $a_1 = \frac{1}{1} = 1, a_2 = \frac{1}{2}, ...$\\
\end{frame}

\begin{frame}{Geometrisk Progresjon (Geometric Progression)}
    \pause
    \begin{block}{Definisjon}
        \begin{itemize}
            \item Følge som har form $a, ar, ar^2, ..., ar^n, ...$
            \item $a$ kalles for \textit{startterm} (initial term) og $r$ kalles for \textit{fellesforhold} (common ratio)
        \end{itemize}
    \end{block}
    \pause
    Eksempler:\\
    \begin{itemize}
        \item Følge $\{b_n\}$ der $b_n = 2 \cdot 5^n$, altså $b_1 = 2 \cdot 5^0 = 2, b_2 = 10, b_3 = 50, ...$
        \item Følge $\{c_n\}$ der $c_n = 6 \cdot \big( \frac{1}{3} \big)^n$, altså $c_1 = 6 \cdot \big( \frac{1}{3} \big)^0 = 6, c_2 = 2, c_3 = \frac{2}{3}, ...$
    \end{itemize}
\end{frame}

\begin{frame}{Aritmetisk Progresjon (Arithmetic Progression)}
    \pause
    \begin{block}{Definisjon}
        \begin{itemize}
            \item følge som har form $a, a + d, a + 2d, ..., a + nd, ...$
            \item $a$ kalles for \textit{startterm} og $d$ kalles for \textit{fellesdifferanse} (common difference)
        \end{itemize}
    \end{block}
    \pause
    Eksempler:\\
    \begin{itemize}
        \item Følge $\{s_n\}$ der $s_n = -1 + 4n$, altså $s_1 = -1 + 4 \cdot 0 = -1, s_2 = 3, s_3 = 7, ...$
        \item Følge $\{t_n\}$ der $t_n = 7 - 3n$, altså $t_1 = 7 - 3 \cdot 0 = 7, t_2 = 4, t_3 = 1, ...$
    \end{itemize}
\end{frame}

\begin{frame}{Rekurrensrelasjoner (Recurrence Relations)}
    \begin{block}{Definisjon}
        \begin{itemize}
            \item uttrykker $a_n$ med forrige termer i følgen, dvs. noen av $a_0, a_1, ..., a_{n-1}$
            \item en følge kalles for løsning av en rekurrensrelasjon hvis det tilfredstiller alle kravene.
        \end{itemize}
    \end{block}
    \pause
    Eksempler:\\
    \begin{itemize}
        \item La $\{a_n\}$ være løsningen for rekurrensrelasjonen $a_n = a_{n-1} + 3$ for $n \in \mathbb{N}$ og $a_0 = 2$. Hva er $a_1$, $a_2$ og $a_3$?
        \pause
        \item Svar: $a_1 = 5$, $a_2 = 8$ og $a_3 = 11$, ikke noe overraskelse her
        \item Fun fact: Kravet $a_0 = 2$ kalles for \textit{startbetingelsen} (initial condition)
    \end{itemize}
\end{frame}

\begin{frame}{Summeringer (Summations)}
    $$a_m + a_{m+1} ? ... + a_n = \displaystyle\sum_{j=m}^{n} a_j$$\\
    $j$ heter da \textit{summeringsindeks}, $m$ heter \textit{nedre grense} og $n$ heter \textit{øvre grense}\\
    \pause
    \begin{block}{Geometrisk rekke (Geometric series)}
        For $a \in \mathbb{R}$ og $r \in \mathbb{R}\setminus\{0\}$
        $$\displaystyle\sum_{j=0}^{n} ar^j = \begin{cases}
            \frac{ar^{n+1} - a}{r - 1} & \quad \text{if } r \neq 0\\
            (n+1)a                     & \quad \text{if } r = 0
          \end{cases}$$
    \end{block}
\end{frame}

