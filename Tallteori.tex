\section{Tallteori}
\subsection*{Div og mod}

\begin{frame}{Divisjon og Modulær aritmetikk}
\begin{block}{Delelighet $a|b$ ($a$ deler $b$)}
$a$ kan dele $b$ uten rest\\
$a|b$ er det samme som $\frac{b}{a}=c$ eller $b=a\cdot c$ med $c \in \matbb{Z}$\\
Eksempel: $3|12$ eller $\frac{12}{3} = 4$ eller $12=3\cdot 4$
\end{block}
\pause
\begin{block}{Modulo (Klokkearitmetikk)}
$a\, mod\, b$ gir ut resten av heltall divisjon av $\frac{a}{b}$ ($a\%b$ i programmeringsspråk)
$a\, mod\, b\, =\, r$ kalles \textit{remainder}
Eksempel: $17\, mod\, 5\, = 2$ fordi $17=3\cdot 5+2$ 
\end{block}
\end{frame}

\begin{frame}[fragile]{Algoritme for divisjon /modulo}
\begin{itemize}[<+->]
\item $d=q\cdot a + r$ med
\item $q=\floor{\frac{d}{a}}$ og $r=d\, mod\, a$
\item Eksempel: $q=\floor{\frac{17}{5}} = \floor{3.4}=3$
\item $17=3\cdot 5 + r$ $\iff$ $17=15+r$ $\iff$ $r=2$
\end{itemize}
\end{frame}

\begin{frame}{Modulo regneregler}
\begin{block}{Kongruens $\equiv$}
\begin{itemize}
\item $a \equiv b\, (mod\, m)$: a og b kongruent i forhold til mod m
\item $a \equiv b\, (mod\, m)$ betyr $a\, mod\, m=b\, mod\, m$
\item vi skriver $[a]_m := a (mod\, m)$
\item Eksempel: $8\, \equiv 3\, (mod\, 5) \equiv [3]_5$ betyr $[8]_5=3=[3]_5$ 
\end{itemize}
\end{block}
\pause
\begin{itemize}[<+->]
\item Addisjon: $[a+b]_m = [[a]_m + [b]_m]_m$
\item $[8+21]_6 = [[8]_6 + [21]_6]_6 = [2 + 3]_6 = [5]_6 = 5$
\item Multiplikasjon: $[a\cdot b]_m = [[a]_m \cdot [b]_m]_m$
\item $[8 \cdot 21]_6 = [[8]_6 \cdot [21]_6]_6 = [2 \cdot 3]_6 = [6]_6 = 0$
\end{itemize}
\end{frame}

\begin{frame}{}
\begin{exampleblock}{Eksempel}
\begin{itemize}
\item $x \equiv 3\,(mod\, 5)$ eller $[x]_5=3$
\item $y \equiv 4\,(mod\, 5)$ eller $[y]_5=4$
\item Finn løsningen: $(3\cdot x+2\cdot y^2)\, mod\,5$
\end{itemize}
\end{exampleblock}
\pause
\medskip

$[3\cdot x+2\cdot y^2]_5=[[3\cdot x]_5+[2\cdot y^2]_5]_5$\\

$[3\cdot x]_5=[[3]_5\cdot [x]_5]_5=[3\cdot 3]_5=[9]_5=4$\\
$[2\cdot y^2]_5=[[2]_5\cdot [y\cdot y]_5]_5=[[2]_5\cdot [y]_5\cdot[y]_5]_5=[2\cdot 4\cdot 4]_5=[32]_5=2$\\

$[[3\cdot x]_5+[2\cdot y^2]_5]_5=[4+2]_5=[6]_5=1$
\end{frame}

\begin{frame}[fragile]{Modulo subtraksjon og divisjon}
       Vi vet at vi har addisjon og multiplikasjon, men hva er med subtrasjon og divisjon?\\

Substraksjon:\\
$[6-3]_8=[3]_8$\\
$[3-6]_8$?\\
\pause
$[3-6]_8=[-3]_8=[0-3]_8=[8-3]_8=[5]_8=5$\\

Subtraksjon fungerer også for modulo.  
\end{frame}

\begin{frame}{Modulo subtraksjon og divisjon}
Vi vet at vi har addisjon og multiplikasjon, men hva med subtrasjon og divisjon?\\

Divisjon:\\
$[6/3]_8=[2]_8$? \pause ja, fordi $[2\cdot 3]_8=[6]_8$\\
$[3/6]_8$?\\
\pause
Nei, noen ganger fungerer det, noen ganger fungerer det ikke.\\

Vi kan ikke alltid dele!
\end{frame}

\subsection*{Tallsystem}
\begin{frame}
\begin{block}{Tallsystem}
En representasjon av tall med forskjellige tegn med en base
\end{block}
\pause

\medskip

\begin{table}[]
\centering
\label{tab:tallsystemer}
\begin{tabular}{l|l|l|l|l}
Navn & Sifre & 5 & 11 & 34 \\ \hline
Desimal (b=10) & 0-9 & 5& 11 & 34 \\
Binær (b=2) & 0-1 & 101 & 1011 & 100010 \\
Octal (b=8) & 0-7 & 5&  13 & 42\\
Hexadesimal (b=16) & 0-9,a-f& 5& B& 22\\
base=13 & 0-9,a-c & 5& B& 28
\end{tabular}
\caption{Eksempler på forskjellige tallsystemer}
\end{table}
\end{frame}

\subsection*{Konvertering av tallsystem}
\begin{frame}[fragile]{Desimal til base b}
\begin{columns}
    \begin{column}{0.30\textwidth}
    \begin{block}{pseudokode}
            tall n til base b:\\
            next digit = $n\%b$\\
            $n=\floor{\frac{n}{b}}$\\
            forsette med det til $n=0$
    \end{block}
    \end{column}
 	\pause
    \begin{column}{0.66\textwidth}
\begin{table}
\begin{tabular}{r|c|r}
n & nextDigit & output \\ \hline
22 & & 0 \\
11 & 0 & 0\\
5 & 1 & 10\\
2 & 1 & 110\\
1 & 0 & 0110\\
0 & 1 & 10110
\end{tabular}
\caption{Eksempel for dec til base 2}
\end{table}
 	\end{column}
 	\end{columns}
\end{frame}

\begin{frame}[fragile]{Base b til desimal}
\begin{columns}
    \begin{column}{0.30\textwidth}
\begin{block}{pseudokode}
    tall n og base b\\
    $sum=0$; $index=0$\\
    starter med først siffer $s$:\\
    $sum+=base^{index}\cdot s$\\
    $index+=1$\\
    forsette med hver siffer $s$
\end{block}
 	\end{column}
 	\pause
    \begin{column}{0.66\textwidth}
\begin{table}
%\begin{tabular}{r|r|r}
%toAdd & n & sum \\ \hline
% & 10110 & 0\\
%0 & 1011 & 0\\
%2 & 101 & 2\\
%4 & 10 & 6\\
%0 & 1 & 6\\
%16 & 0 & 22
%\end{tabular}
\begin{tabular}{c|r|r|r|r|r}
    tall & 1 & 0 & 1 & 1 & 0 \\ \hline
    base & 16 & 8 & 4 & 2 & 1 \\ \hline
    produkt & 16 & 0 & 4 & 2 & 0 
\end{tabular}
\caption{Eksempel for 2 til dec}
\end{table}
\begin{center}
$16+0+4+2+0=22$    
\end{center}

 	\end{column}
 	\end{columns}
\end{frame}

\subsection*{Primtall, Greatest common divisor, Least common multiple}
\begin{frame}
\begin{block}{Primtall}
Et tall som bare kan deles av seg selv og 1\\
Eksempler: 2,3,5,7,11,13,...
\end{block}
\pause
\begin{block}{Største felles faktor (/Greatest common divisor)}
$gcd(a,b) := $det største tallet som deler både a og b\\
Eksempel: $gcd(4,6)=2$\\
Co-prime: a og b er co-prime dersom $gcd(a,b)=1$
\end{block}
\pause
\begin{block}{Laveste felles multiplum (/Least common multiple)}
$lcm(a,b) := $ det minste tallet som kan deles av både a og b\\
Eksempel: $lcm(4,6)=12$\\
hvis $a\cdot b=gcd(a,b)\cdot lcm(a,b)$\\
$gcd(a,b) = 1$ $\rightarrow lcm(a,b) = a\cdot b$
\end{block}
\end{frame}

\subsection*{Euklids algoritme}
\begin{frame}[fragile]{Euklids algoritme}
\begin{columns}
    \begin{column}{0.45\textwidth}
\begin{minted}[fontsize=\scriptsize]{python}
def gcd(a, b):
    while b != 0:
        r = a % b
        a = b
        b = r
    return a
\end{minted}
 	\end{column}
 	\pause
    \begin{column}{0.45\textwidth}
    \begin{center}
       gcd(28,12)\\
       
    $28=2\cdot 12+4$\\
    $12=3\cdot 4+0$\\
    
    gcd(28,12)=4 
    \end{center}
 	\end{column}
\end{columns}


\end{frame}

\begin{frame}[fragile]{}
\begin{block}{Utvidet Euklids algoritme}
Regner ut to parameter $s$ og $t$ slik at $gcd(a,b)$ kan skrives som linærkombinasjon\\
$gcd(a,b)=s\cdot a+t\cdot b$\\
$gcd(12,28)=4=-2\cdot 12 + 1\cdot 28$\medskip

Kan brukes for å finne multiplikativt invers\\
Multiplikativ inverse finnes dersom $gcd(a,b)=1$
\end{block}
\pause

\begin{block}{Finne multiplikativt invers for $a$ med $mod\, m$}
\begin{itemize}
\item Funker bare dersom $gcd(a,m)=1$\\
\item Regn ut linærkombinasjon $gcd(a,b)=s\cdot a+t\cdot b$ med gcd
\item $a\cdot x \equiv 1 (mod\, m)$ er multiplicative inverse
\end{itemize}
\end{block}
\end{frame}

\begin{frame}{Extended Euklids algoritme}
    \begin{columns}
        \begin{column}{0.45\textwidth}
             gcd(26,7)\\
             
             $(26)=3\cdot (7)+(5)$\\
             $(7)=1\cdot (5)+(2)$\\
             $(5)=2\cdot(2)+(1)$\\
             $(2)=2\cdot(1)+(0)$\\
             
             gcd(26,7)=1
        \end{column}
        \pause
        \begin{column}{0.45\textwidth}
            Nå går vi tilbake:\\
            $(5)=2\cdot(2)+1$\\
            
            $\implies 1=(5)-2\cdot (2)$\\
            $=(5)-2\cdot ((7)-(5))$\\
            $=3\cdot (5)-2\cdot(7)$\\
            $=3\cdot ((26)-3\cdot (7))-2\cdot (7)$\\
            $=3\cdot (26)-11\cdot (7)$
        \end{column}
    \end{columns}
\end{frame}


\begin{frame}{Eksempel Multiplikativt Invers}
\begin{itemize}[<+->]
\item Hva er multiplikativt invers av $7\, mod\, 26$? ($a\cdot 7=1\, mod\, 26$)
\item $gcd(a,m) = gcd(7,26)=1$ $\rightarrow$ har multiplikativt invers
\item Linærkombinasjon fra gcd: $3\cdot (26)-11\cdot (7)=1$
\item $[1]_{26}=[3\cdot (26)-11\cdot (7)]_{26}=[-11\cdot 7]_{26}$
\item $a=-11$
\item $[-11]_{26}=[26-11]_{26}=[15]_{26}=15$
\item 15 er inverse av 7 modulo 26

\end{itemize}
\end{frame}

\subsection*{Spørretid}
\begin{frame}{Spørsmål?}
    \begin{figure}
        \centering
        \includegraphics[height = 4.9cm]{images/guillaume1.jpg}
        \caption{Guillaume på Sandviksfjellet}
        \label{fig:guillaume1}
    \end{figure}
\end{frame}

% ===========================

\section{Kryptografi}
\subsection*{Begrep}
\begin{frame}{Symmetrisk og asymmetrisk kryptografi}
\begin{block}{Symmetrisk kryptografi}
\begin{itemize}
\item Det finnes bare \textit{én} nøkkel, som begge personer bruker
\item Brukes for både kryptering og dekryptering
\end{itemize}
\end{block}
\pause
\begin{block}{Asymmetrisk kryptografi}
\begin{itemize}
\item Hver person har \textit{to} nøkler: Privat og offentlig
\item Kryptering med offentlig nøkkel av den andre personen
\item Dekryptering med privat nøkkel
\item Eksempel: RSA
\end{itemize}
\end{block}
\end{frame}

\begin{frame}{Symmetrisk kryptografi}
    $f(c)=[c+key]_{26}$\\
    med $key=2$\\
    $f(ZEBRA)=BGDTC$\\
    
    $f^{-1}(d)=[d-key]_{26}$\\
    $f^{-1}(BGDTC)=ZEBRA$
\end{frame}

\subsection*{RSA}
\begin{frame}{RSA}
\begin{itemize}[<+->]
\item Asymmetrisk kryptering med to nøkler for hver deltaker
\item Kryptering
	\begin{itemize}
	\item Offentlig nøkkel for kryptering (n,e)
	\item Privat nøkkel for dekryptering d
	\end{itemize}
 \item $d$ er inverset av $e$ modulo $(p-1)\cdot (q-1)$ 
 \item med $p\cdot q=n$ og p,q er primtall
\end{itemize}
\end{frame}

\begin{frame}{}
    \begin{columns}
        \begin{column}{0.45 \textwidth}
            Instruksjon:\\
            velg to primtall $p,q$\\
            $n=p\cdot q$
            Velg $e$ med $2<e<\phi(n)=(p-1)\cdot (q-1)$ og $gcd(e,\phi(n))=1$\\
            Finn $d=[e^{-1}]_{\phi(n)}$\\
            Gir ut bare offentlig nøkkel $(n,e)$
        \end{column}
        \begin{column}{0.45 \textwidth}
            Eksempel:\\
            $p=7,\,q=13$\\
            $n=7\cdot 13=91$\\
            $e$ mellom $2$ og $6\cdot 12=72$ og $gcd(e,72)=1$\\
            Vi prøve 23 på neste slide:
        \end{column}
    \end{columns}
\end{frame}

\begin{frame}{Finn $d$ og $e$}
    \begin{columns}
        \begin{column}{0.45 \textwidth}
            $gcd(72,23):$\\
            $(72)=3\cdot (23)+(3)$\\
            $(23)=7\cdot(3)+(2)$\\
            $(3)=1\cdot (2)+(1)$\\
            $(2)=2\cdot (1)+(0)$\\
            gcd(72,23)=1\\
            Vi har $e=23$
        \end{column}
        \begin{column}{0.45 \textwidth}
            Når finner vi $d=[e^{-1}]_{72}$:\\
            $1=(3)-1\cdot (2)$\\
            $=(3)-((23)-7\cdot (3))$\\
            $=8\cdot (3)-(23)$\\
            $=8\cdot ((72)-3\cdot (23))-(23)$\\
            $=8\cdot (72)-25\cdot (23)$\\
            $\implies d=[-25]_{72}=[72-25]_{72}=47$
        \end{column}
    \end{columns}
\end{frame}

\begin{frame}{Kryptering}
    \begin{column}{0.45 \textwidth}
        Kryptering av blokk M:\\
        $C=[M^{e}]_{n}$\\
        $M=42$; $n=91, e=23$\\
        men $42^{23}=2.16\times 10^{37}$\\
        \pause
        $e=23=10111_2$\\
        
        $[42^{23}]_{91}=[42^{1}]_{91}\cdot[42^{2}]_{91}\cdot[42^{4}]_{91}\cdot [42^{16}]_{91}$
        \end{column}
        \pause
        \begin{column}{0.45 \textwidth}
         $[42^1]_{91}=42$\\
         $[42^2]_{91}=[1764]_{91}=35$\\
         $[42^4]_{91}=[35^2]_{91}=[1225]_{91}=42$\\
         $[42^8]_{91}=[42^2]_{91}=35$\\
         $[42^{16}]_{91}=[35^2]_{91}=42$\\
         
         $[42^{23}]_{91}=[42\cdot 35\cdot 42\cdot 42]_{91}=[2593080]_{91}=35$
        \end{column}
\end{frame}

\begin{frame}{Dekryptering}
    \begin{columns}
        \begin{column}{0.45 \textwidth}
        Dekryptering av blokk C:\\
        $M=[C^d]_{n}$\\
        $C=35$; $n=91, d=47$\\
        $d=47=101111_2$\\
        \end{column}

        \begin{column}{0.45 \textwidth}
        $[35^1]_{91}=35$\\
        $[35^2]_{91}=42$\\
        $[35^4]_{91}=35$\\
        $[35^8]_{91}=42$\\
        $[35^{16}]_{91}=35$\\
        $[35^{32}]_{91}=42$\\
        $[35^{47}]_{91}=[35\cdot 42\cdot 35 \cdot 42 \cdot 42]_{91}=[35^2\cdot 42^3]_{91}=[42^4]_{91}=42$
        \end{column}
    \end{columns}
\end{frame}

\subsection*{Spørretid}
\begin{frame}{Spørsmål?}
    \begin{figure}
        \centering
        \includegraphics[height = 4.9cm]{images/guillaume8.jpg}
        \caption{Guillaume på Lyderhorn}
        \label{fig:guillaume8}
    \end{figure}
\end{frame}
